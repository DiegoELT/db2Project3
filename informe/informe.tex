\documentclass{article}

\renewcommand\refname{Referencias}

\usepackage{graphicx}

\title{CouchDB: Investigación e implemetanción demostrativa de una base de datos NoSQL}
\author{Sebastián Hurtado \and Diego Linares \and Piero Marini}
\date{28 de Noviembre del 2019}

\begin{document}
    \maketitle  
    \section{Introducción}
        El objetivo de la presente investigación es entender el funcionamiento de una base de datos NoSQL y realizar una demostración de esta con una colección de datos. Actualmente, la data que resulta del uso en particular de aplicaciones web, redes sociales, etc. se encuentra en un nivel bajo de estructuración. Como resultado de ello, el uso de base de datos relacionales puede no ser el más adecuado para trabajar con este tipo de data \cite{sangeeta}. En este caso, exploraremos una de las soluciones NoSQL, que provee soporte ACID y permite trabajar con data no estructurada en formato JSON, CouchDB. \\
        El siguiente artículo se va a encontrar dividido de la siguiente manera: primero se definirá CouchDB y se comparará frente a otras soluciones de este tipo, observando en que casos conviene el uso del primero. A continuación explicaremos más a detalle las características del modelo de datos y la arquitectura de almacenamiento distribuido. Finalmente se hará una demostración de su uso en una colección de datos del servicio web Yelp, y se realizarán conclusiones en base a la misma. 
        \subsection{Definición y Propósito}

        \subsection{Cuadro comparativo con otras BD NoSQL}
        \subsection{Escenarios de uso}
    \section{Características}
        \subsection{Modelo de datos}
            \subsubsection{Inserciones}
            \subsubsection{Actualizaciones}
            \subsubsection{Búsqueda}
            \subsubsection{Indexación} 
        \subsection{Arquitectura de almacenamiento distribuido}
            \subsubsection{Fragmentación, Asignación, Replicación}
            \subsubsection{Procesamiento de consultas distribuidas}
    \section{Implementación Demostrativa}
        \subsection{Carga de una colección de datos}
        \subsection{Consulta de datos de forma distribuida}
    \section{Conclusiones}

    \newpage

    \bibliographystyle{unsrt}
    \bibliography{bibliography}
\end{document}